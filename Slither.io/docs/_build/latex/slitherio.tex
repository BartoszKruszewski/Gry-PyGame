%% Generated by Sphinx.
\def\sphinxdocclass{report}
\documentclass[letterpaper,10pt,polish]{sphinxmanual}
\ifdefined\pdfpxdimen
   \let\sphinxpxdimen\pdfpxdimen\else\newdimen\sphinxpxdimen
\fi \sphinxpxdimen=.75bp\relax
\ifdefined\pdfimageresolution
    \pdfimageresolution= \numexpr \dimexpr1in\relax/\sphinxpxdimen\relax
\fi
%% let collapsible pdf bookmarks panel have high depth per default
\PassOptionsToPackage{bookmarksdepth=5}{hyperref}

\PassOptionsToPackage{booktabs}{sphinx}
\PassOptionsToPackage{colorrows}{sphinx}

\PassOptionsToPackage{warn}{textcomp}
\usepackage[utf8]{inputenc}
\ifdefined\DeclareUnicodeCharacter
% support both utf8 and utf8x syntaxes
  \ifdefined\DeclareUnicodeCharacterAsOptional
    \def\sphinxDUC#1{\DeclareUnicodeCharacter{"#1}}
  \else
    \let\sphinxDUC\DeclareUnicodeCharacter
  \fi
  \sphinxDUC{00A0}{\nobreakspace}
  \sphinxDUC{2500}{\sphinxunichar{2500}}
  \sphinxDUC{2502}{\sphinxunichar{2502}}
  \sphinxDUC{2514}{\sphinxunichar{2514}}
  \sphinxDUC{251C}{\sphinxunichar{251C}}
  \sphinxDUC{2572}{\textbackslash}
\fi
\usepackage{cmap}
\usepackage[T1]{fontenc}
\usepackage{amsmath,amssymb,amstext}
\usepackage{babel}



\usepackage{tgtermes}
\usepackage{tgheros}
\renewcommand{\ttdefault}{txtt}



\usepackage[Sonny]{fncychap}
\ChNameVar{\Large\normalfont\sffamily}
\ChTitleVar{\Large\normalfont\sffamily}
\usepackage{sphinx}

\fvset{fontsize=auto}
\usepackage{geometry}


% Include hyperref last.
\usepackage{hyperref}
% Fix anchor placement for figures with captions.
\usepackage{hypcap}% it must be loaded after hyperref.
% Set up styles of URL: it should be placed after hyperref.
\urlstyle{same}

\addto\captionspolish{\renewcommand{\contentsname}{Contents:}}

\usepackage{sphinxmessages}
\setcounter{tocdepth}{1}



\title{Slither.io}
\date{04 cze 2023}
\release{1.0.0}
\author{Bartosz Kruszewski}
\newcommand{\sphinxlogo}{\vbox{}}
\renewcommand{\releasename}{Wydanie}
\makeindex
\begin{document}

\ifdefined\shorthandoff
  \ifnum\catcode`\=\string=\active\shorthandoff{=}\fi
  \ifnum\catcode`\"=\active\shorthandoff{"}\fi
\fi

\pagestyle{empty}
\sphinxmaketitle
\pagestyle{plain}
\sphinxtableofcontents
\pagestyle{normal}
\phantomsection\label{\detokenize{index::doc}}


\sphinxstepscope


\chapter{slitherio}
\label{\detokenize{modules:slitherio}}\label{\detokenize{modules::doc}}
\sphinxstepscope


\section{CONST module}
\label{\detokenize{const:module-const}}\label{\detokenize{const:const-module}}\label{\detokenize{const::doc}}\index{moduł@\spxentry{moduł}!const@\spxentry{const}}\index{const@\spxentry{const}!moduł@\spxentry{moduł}}
\sphinxAtStartPar
Moduł zawierający stałe wykorzystywane w grze.
\index{AI\_ATTACK\_FOCUS\_DISTANCE (w module const)@\spxentry{AI\_ATTACK\_FOCUS\_DISTANCE}\spxextra{w module const}}

\begin{fulllineitems}
\phantomsection\label{\detokenize{const:const.AI_ATTACK_FOCUS_DISTANCE}}
\pysigstartsignatures
\pysigline{\sphinxbfcode{\sphinxupquote{AI\_ATTACK\_FOCUS\_DISTANCE}}\sphinxbfcode{\sphinxupquote{\DUrole{w,w}{  }\DUrole{p,p}{=}\DUrole{w,w}{  }100}}}
\pysigstopsignatures
\sphinxAtStartPar
Zasięg w jakim AI zaczyna atakować inne węże.

\end{fulllineitems}

\index{AI\_SAFETY\_DISTANCE (w module const)@\spxentry{AI\_SAFETY\_DISTANCE}\spxextra{w module const}}

\begin{fulllineitems}
\phantomsection\label{\detokenize{const:const.AI_SAFETY_DISTANCE}}
\pysigstartsignatures
\pysigline{\sphinxbfcode{\sphinxupquote{AI\_SAFETY\_DISTANCE}}\sphinxbfcode{\sphinxupquote{\DUrole{w,w}{  }\DUrole{p,p}{=}\DUrole{w,w}{  }20}}}
\pysigstopsignatures
\sphinxAtStartPar
Odległość od węża po przekroczeniu, której AI będzie zawracać unikając
kolizji

\end{fulllineitems}

\index{AI\_SPEED\_UP\_DISTANCE (w module const)@\spxentry{AI\_SPEED\_UP\_DISTANCE}\spxextra{w module const}}

\begin{fulllineitems}
\phantomsection\label{\detokenize{const:const.AI_SPEED_UP_DISTANCE}}
\pysigstartsignatures
\pysigline{\sphinxbfcode{\sphinxupquote{AI\_SPEED\_UP\_DISTANCE}}\sphinxbfcode{\sphinxupquote{\DUrole{w,w}{  }\DUrole{p,p}{=}\DUrole{w,w}{  }200}}}
\pysigstopsignatures
\sphinxAtStartPar
Dystans, po którego przekroczeniu AI przyśpiesza w kierunku celu.

\end{fulllineitems}

\index{AI\_UPDATE\_TIME (w module const)@\spxentry{AI\_UPDATE\_TIME}\spxextra{w module const}}

\begin{fulllineitems}
\phantomsection\label{\detokenize{const:const.AI_UPDATE_TIME}}
\pysigstartsignatures
\pysigline{\sphinxbfcode{\sphinxupquote{AI\_UPDATE\_TIME}}\sphinxbfcode{\sphinxupquote{\DUrole{w,w}{  }\DUrole{p,p}{=}\DUrole{w,w}{  }20}}}
\pysigstopsignatures
\sphinxAtStartPar
Odstęp czasu pomiędzy odświeżeniem AI.

\end{fulllineitems}

\index{BACKGROUND\_COLOR (w module const)@\spxentry{BACKGROUND\_COLOR}\spxextra{w module const}}

\begin{fulllineitems}
\phantomsection\label{\detokenize{const:const.BACKGROUND_COLOR}}
\pysigstartsignatures
\pysigline{\sphinxbfcode{\sphinxupquote{BACKGROUND\_COLOR}}\sphinxbfcode{\sphinxupquote{\DUrole{w,w}{  }\DUrole{p,p}{=}\DUrole{w,w}{  }(40, 40, 40)}}}
\pysigstopsignatures
\sphinxAtStartPar
Kolor tła.

\end{fulllineitems}

\index{ENEMY\_AMOUNT (w module const)@\spxentry{ENEMY\_AMOUNT}\spxextra{w module const}}

\begin{fulllineitems}
\phantomsection\label{\detokenize{const:const.ENEMY_AMOUNT}}
\pysigstartsignatures
\pysigline{\sphinxbfcode{\sphinxupquote{ENEMY\_AMOUNT}}\sphinxbfcode{\sphinxupquote{\DUrole{w,w}{  }\DUrole{p,p}{=}\DUrole{w,w}{  }25}}}
\pysigstopsignatures
\sphinxAtStartPar
Ilość przeciwników na mapie.

\end{fulllineitems}

\index{FOOD\_COLORS (w module const)@\spxentry{FOOD\_COLORS}\spxextra{w module const}}

\begin{fulllineitems}
\phantomsection\label{\detokenize{const:const.FOOD_COLORS}}
\pysigstartsignatures
\pysigline{\sphinxbfcode{\sphinxupquote{FOOD\_COLORS}}\sphinxbfcode{\sphinxupquote{\DUrole{w,w}{  }\DUrole{p,p}{=}\DUrole{w,w}{  }((0, 255, 0), (0, 200, 20), (20, 140, 10), (40, 255, 10), (5, 230, 80))}}}
\pysigstopsignatures
\sphinxAtStartPar
Kolory jedzenia.

\end{fulllineitems}

\index{FOOD\_LIMIT (w module const)@\spxentry{FOOD\_LIMIT}\spxextra{w module const}}

\begin{fulllineitems}
\phantomsection\label{\detokenize{const:const.FOOD_LIMIT}}
\pysigstartsignatures
\pysigline{\sphinxbfcode{\sphinxupquote{FOOD\_LIMIT}}\sphinxbfcode{\sphinxupquote{\DUrole{w,w}{  }\DUrole{p,p}{=}\DUrole{w,w}{  }150}}}
\pysigstopsignatures
\sphinxAtStartPar
Maksymalna ilość jedzenia na mapie.

\end{fulllineitems}

\index{FOOD\_SPAWN\_TIME (w module const)@\spxentry{FOOD\_SPAWN\_TIME}\spxextra{w module const}}

\begin{fulllineitems}
\phantomsection\label{\detokenize{const:const.FOOD_SPAWN_TIME}}
\pysigstartsignatures
\pysigline{\sphinxbfcode{\sphinxupquote{FOOD\_SPAWN\_TIME}}\sphinxbfcode{\sphinxupquote{\DUrole{w,w}{  }\DUrole{p,p}{=}\DUrole{w,w}{  }20}}}
\pysigstopsignatures
\sphinxAtStartPar
Czas pojawiania się jedzenia.

\end{fulllineitems}

\index{FOOD\_TO\_GROW (w module const)@\spxentry{FOOD\_TO\_GROW}\spxextra{w module const}}

\begin{fulllineitems}
\phantomsection\label{\detokenize{const:const.FOOD_TO_GROW}}
\pysigstartsignatures
\pysigline{\sphinxbfcode{\sphinxupquote{FOOD\_TO\_GROW}}\sphinxbfcode{\sphinxupquote{\DUrole{w,w}{  }\DUrole{p,p}{=}\DUrole{w,w}{  }5}}}
\pysigstopsignatures
\sphinxAtStartPar
Ilość jedzenia, którą musi zjeść wąż, żeby urosnąć o jedną „komórkę”
więcej

\end{fulllineitems}

\index{FRAMERATE (w module const)@\spxentry{FRAMERATE}\spxextra{w module const}}

\begin{fulllineitems}
\phantomsection\label{\detokenize{const:const.FRAMERATE}}
\pysigstartsignatures
\pysigline{\sphinxbfcode{\sphinxupquote{FRAMERATE}}\sphinxbfcode{\sphinxupquote{\DUrole{w,w}{  }\DUrole{p,p}{=}\DUrole{w,w}{  }60}}}
\pysigstopsignatures
\sphinxAtStartPar
Maksymalna liczba klatek na sekundę.

\end{fulllineitems}

\index{MAP\_SIZE (w module const)@\spxentry{MAP\_SIZE}\spxextra{w module const}}

\begin{fulllineitems}
\phantomsection\label{\detokenize{const:const.MAP_SIZE}}
\pysigstartsignatures
\pysigline{\sphinxbfcode{\sphinxupquote{MAP\_SIZE}}\sphinxbfcode{\sphinxupquote{\DUrole{w,w}{  }\DUrole{p,p}{=}\DUrole{w,w}{  }1000}}}
\pysigstopsignatures
\sphinxAtStartPar
Wielkość mapy.

\end{fulllineitems}

\index{MINIMUM\_DRAW\_SCREEN\_SIZE (w module const)@\spxentry{MINIMUM\_DRAW\_SCREEN\_SIZE}\spxextra{w module const}}

\begin{fulllineitems}
\phantomsection\label{\detokenize{const:const.MINIMUM_DRAW_SCREEN_SIZE}}
\pysigstartsignatures
\pysigline{\sphinxbfcode{\sphinxupquote{MINIMUM\_DRAW\_SCREEN\_SIZE}}\sphinxbfcode{\sphinxupquote{\DUrole{w,w}{  }\DUrole{p,p}{=}\DUrole{w,w}{  }(180, 90)}}}
\pysigstopsignatures
\sphinxAtStartPar
Minimalna wielkość płaszczyzny do rysowania.

\end{fulllineitems}

\index{PARTICLE\_AMOUNT (w module const)@\spxentry{PARTICLE\_AMOUNT}\spxextra{w module const}}

\begin{fulllineitems}
\phantomsection\label{\detokenize{const:const.PARTICLE_AMOUNT}}
\pysigstartsignatures
\pysigline{\sphinxbfcode{\sphinxupquote{PARTICLE\_AMOUNT}}\sphinxbfcode{\sphinxupquote{\DUrole{w,w}{  }\DUrole{p,p}{=}\DUrole{w,w}{  }5}}}
\pysigstopsignatures
\sphinxAtStartPar
Ilość cząsteczek generowanych przy każdej „komórce” węża.

\end{fulllineitems}

\index{PARTICLE\_COLORS (w module const)@\spxentry{PARTICLE\_COLORS}\spxextra{w module const}}

\begin{fulllineitems}
\phantomsection\label{\detokenize{const:const.PARTICLE_COLORS}}
\pysigstartsignatures
\pysigline{\sphinxbfcode{\sphinxupquote{PARTICLE\_COLORS}}\sphinxbfcode{\sphinxupquote{\DUrole{w,w}{  }\DUrole{p,p}{=}\DUrole{w,w}{  }((255, 255, 255, 255), (220, 220, 220, 255), (220, 0, 220, 255), (110, 0, 110, 255))}}}
\pysigstopsignatures
\sphinxAtStartPar
Kolory cząsteczek.

\end{fulllineitems}

\index{PARTICLE\_SIZE (w module const)@\spxentry{PARTICLE\_SIZE}\spxextra{w module const}}

\begin{fulllineitems}
\phantomsection\label{\detokenize{const:const.PARTICLE_SIZE}}
\pysigstartsignatures
\pysigline{\sphinxbfcode{\sphinxupquote{PARTICLE\_SIZE}}\sphinxbfcode{\sphinxupquote{\DUrole{w,w}{  }\DUrole{p,p}{=}\DUrole{w,w}{  }(1, 4)}}}
\pysigstopsignatures
\sphinxAtStartPar
Wielkość cząsteczek.

\end{fulllineitems}

\index{PARTICLE\_SPEED (w module const)@\spxentry{PARTICLE\_SPEED}\spxextra{w module const}}

\begin{fulllineitems}
\phantomsection\label{\detokenize{const:const.PARTICLE_SPEED}}
\pysigstartsignatures
\pysigline{\sphinxbfcode{\sphinxupquote{PARTICLE\_SPEED}}\sphinxbfcode{\sphinxupquote{\DUrole{w,w}{  }\DUrole{p,p}{=}\DUrole{w,w}{  }0.5}}}
\pysigstopsignatures
\sphinxAtStartPar
Prędkość cząsteczek.

\end{fulllineitems}

\index{PARTICLE\_TIME (w module const)@\spxentry{PARTICLE\_TIME}\spxextra{w module const}}

\begin{fulllineitems}
\phantomsection\label{\detokenize{const:const.PARTICLE_TIME}}
\pysigstartsignatures
\pysigline{\sphinxbfcode{\sphinxupquote{PARTICLE\_TIME}}\sphinxbfcode{\sphinxupquote{\DUrole{w,w}{  }\DUrole{p,p}{=}\DUrole{w,w}{  }500}}}
\pysigstopsignatures
\sphinxAtStartPar
Czas trwania cząsteczek.

\end{fulllineitems}

\index{SCREEN\_SIZE (w module const)@\spxentry{SCREEN\_SIZE}\spxextra{w module const}}

\begin{fulllineitems}
\phantomsection\label{\detokenize{const:const.SCREEN_SIZE}}
\pysigstartsignatures
\pysigline{\sphinxbfcode{\sphinxupquote{SCREEN\_SIZE}}\sphinxbfcode{\sphinxupquote{\DUrole{w,w}{  }\DUrole{p,p}{=}\DUrole{w,w}{  }(1280, 720)}}}
\pysigstopsignatures
\sphinxAtStartPar
Wielkość okna gry.

\end{fulllineitems}

\index{SNAKE\_ACCELERATION (w module const)@\spxentry{SNAKE\_ACCELERATION}\spxextra{w module const}}

\begin{fulllineitems}
\phantomsection\label{\detokenize{const:const.SNAKE_ACCELERATION}}
\pysigstartsignatures
\pysigline{\sphinxbfcode{\sphinxupquote{SNAKE\_ACCELERATION}}\sphinxbfcode{\sphinxupquote{\DUrole{w,w}{  }\DUrole{p,p}{=}\DUrole{w,w}{  }0.06}}}
\pysigstopsignatures
\sphinxAtStartPar
Przyśpieszenie węża.

\end{fulllineitems}

\index{SNAKE\_BASE\_SPEED (w module const)@\spxentry{SNAKE\_BASE\_SPEED}\spxextra{w module const}}

\begin{fulllineitems}
\phantomsection\label{\detokenize{const:const.SNAKE_BASE_SPEED}}
\pysigstartsignatures
\pysigline{\sphinxbfcode{\sphinxupquote{SNAKE\_BASE\_SPEED}}\sphinxbfcode{\sphinxupquote{\DUrole{w,w}{  }\DUrole{p,p}{=}\DUrole{w,w}{  }1}}}
\pysigstopsignatures
\sphinxAtStartPar
Początkowa prędkość węża.

\end{fulllineitems}

\index{SNAKE\_BOOST\_SPEED (w module const)@\spxentry{SNAKE\_BOOST\_SPEED}\spxextra{w module const}}

\begin{fulllineitems}
\phantomsection\label{\detokenize{const:const.SNAKE_BOOST_SPEED}}
\pysigstartsignatures
\pysigline{\sphinxbfcode{\sphinxupquote{SNAKE\_BOOST\_SPEED}}\sphinxbfcode{\sphinxupquote{\DUrole{w,w}{  }\DUrole{p,p}{=}\DUrole{w,w}{  }2}}}
\pysigstopsignatures
\sphinxAtStartPar
Prędkość węża przy przyśpieszeniu.

\end{fulllineitems}

\index{SNAKE\_COLORS (w module const)@\spxentry{SNAKE\_COLORS}\spxextra{w module const}}

\begin{fulllineitems}
\phantomsection\label{\detokenize{const:const.SNAKE_COLORS}}
\pysigstartsignatures
\pysigline{\sphinxbfcode{\sphinxupquote{SNAKE\_COLORS}}\sphinxbfcode{\sphinxupquote{\DUrole{w,w}{  }\DUrole{p,p}{=}\DUrole{w,w}{  }((180, 100, 120), (60, 20, 205), (10, 180, 20), (80, 100, 20), (240, 180, 25), (42, 90, 205), (87, 10, 120), (78, 100, 100))}}}
\pysigstopsignatures
\sphinxAtStartPar
Kolory węży.

\end{fulllineitems}

\index{SNAKE\_FADE (w module const)@\spxentry{SNAKE\_FADE}\spxextra{w module const}}

\begin{fulllineitems}
\phantomsection\label{\detokenize{const:const.SNAKE_FADE}}
\pysigstartsignatures
\pysigline{\sphinxbfcode{\sphinxupquote{SNAKE\_FADE}}\sphinxbfcode{\sphinxupquote{\DUrole{w,w}{  }\DUrole{p,p}{=}\DUrole{w,w}{  }4}}}
\pysigstopsignatures
\sphinxAtStartPar
Siła efektu gradientu na ciele węża.

\end{fulllineitems}

\index{SNAKE\_FOOD\_CONSUMPTION\_SPEED (w module const)@\spxentry{SNAKE\_FOOD\_CONSUMPTION\_SPEED}\spxextra{w module const}}

\begin{fulllineitems}
\phantomsection\label{\detokenize{const:const.SNAKE_FOOD_CONSUMPTION_SPEED}}
\pysigstartsignatures
\pysigline{\sphinxbfcode{\sphinxupquote{SNAKE\_FOOD\_CONSUMPTION\_SPEED}}\sphinxbfcode{\sphinxupquote{\DUrole{w,w}{  }\DUrole{p,p}{=}\DUrole{w,w}{  }0.1}}}
\pysigstopsignatures
\sphinxAtStartPar
szybkość „spalania” jedzenia podczas przyśpieszenia węża.

\end{fulllineitems}

\index{SNAKE\_INTERLUDE (w module const)@\spxentry{SNAKE\_INTERLUDE}\spxextra{w module const}}

\begin{fulllineitems}
\phantomsection\label{\detokenize{const:const.SNAKE_INTERLUDE}}
\pysigstartsignatures
\pysigline{\sphinxbfcode{\sphinxupquote{SNAKE\_INTERLUDE}}\sphinxbfcode{\sphinxupquote{\DUrole{w,w}{  }\DUrole{p,p}{=}\DUrole{w,w}{  }5}}}
\pysigstopsignatures
\sphinxAtStartPar
Odstęp pomiędzy „komórkami” węża.

\end{fulllineitems}

\index{SNAKE\_SPAWN\_TIME (w module const)@\spxentry{SNAKE\_SPAWN\_TIME}\spxextra{w module const}}

\begin{fulllineitems}
\phantomsection\label{\detokenize{const:const.SNAKE_SPAWN_TIME}}
\pysigstartsignatures
\pysigline{\sphinxbfcode{\sphinxupquote{SNAKE\_SPAWN\_TIME}}\sphinxbfcode{\sphinxupquote{\DUrole{w,w}{  }\DUrole{p,p}{=}\DUrole{w,w}{  }20}}}
\pysigstopsignatures
\sphinxAtStartPar
Czas pojawiania się węża.

\end{fulllineitems}

\index{SNAKE\_START\_LENGTH (w module const)@\spxentry{SNAKE\_START\_LENGTH}\spxextra{w module const}}

\begin{fulllineitems}
\phantomsection\label{\detokenize{const:const.SNAKE_START_LENGTH}}
\pysigstartsignatures
\pysigline{\sphinxbfcode{\sphinxupquote{SNAKE\_START\_LENGTH}}\sphinxbfcode{\sphinxupquote{\DUrole{w,w}{  }\DUrole{p,p}{=}\DUrole{w,w}{  }5}}}
\pysigstopsignatures
\sphinxAtStartPar
Początkowa długość węża.

\end{fulllineitems}

\index{SNAKE\_START\_SIZE (w module const)@\spxentry{SNAKE\_START\_SIZE}\spxextra{w module const}}

\begin{fulllineitems}
\phantomsection\label{\detokenize{const:const.SNAKE_START_SIZE}}
\pysigstartsignatures
\pysigline{\sphinxbfcode{\sphinxupquote{SNAKE\_START\_SIZE}}\sphinxbfcode{\sphinxupquote{\DUrole{w,w}{  }\DUrole{p,p}{=}\DUrole{w,w}{  }4}}}
\pysigstopsignatures
\sphinxAtStartPar
Początkowy promień „komórki” węża.

\end{fulllineitems}

\index{START\_DRAW\_SCREEN\_SIZE (w module const)@\spxentry{START\_DRAW\_SCREEN\_SIZE}\spxextra{w module const}}

\begin{fulllineitems}
\phantomsection\label{\detokenize{const:const.START_DRAW_SCREEN_SIZE}}
\pysigstartsignatures
\pysigline{\sphinxbfcode{\sphinxupquote{START\_DRAW\_SCREEN\_SIZE}}\sphinxbfcode{\sphinxupquote{\DUrole{w,w}{  }\DUrole{p,p}{=}\DUrole{w,w}{  }(320, 180)}}}
\pysigstopsignatures
\sphinxAtStartPar
Wielkość płaszczyzny do rysowania.

\end{fulllineitems}

\index{START\_FOOD\_AMOUNT (w module const)@\spxentry{START\_FOOD\_AMOUNT}\spxextra{w module const}}

\begin{fulllineitems}
\phantomsection\label{\detokenize{const:const.START_FOOD_AMOUNT}}
\pysigstartsignatures
\pysigline{\sphinxbfcode{\sphinxupquote{START\_FOOD\_AMOUNT}}\sphinxbfcode{\sphinxupquote{\DUrole{w,w}{  }\DUrole{p,p}{=}\DUrole{w,w}{  }100}}}
\pysigstopsignatures
\sphinxAtStartPar
Początkowa ilość jedzenia na mapie.

\end{fulllineitems}

\index{ZOOM\_SPEED (w module const)@\spxentry{ZOOM\_SPEED}\spxextra{w module const}}

\begin{fulllineitems}
\phantomsection\label{\detokenize{const:const.ZOOM_SPEED}}
\pysigstartsignatures
\pysigline{\sphinxbfcode{\sphinxupquote{ZOOM\_SPEED}}\sphinxbfcode{\sphinxupquote{\DUrole{w,w}{  }\DUrole{p,p}{=}\DUrole{w,w}{  }1.2}}}
\pysigstopsignatures
\sphinxAtStartPar
Prędkość przybliżania mapy.

\end{fulllineitems}


\sphinxstepscope


\section{enemy module}
\label{\detokenize{enemy:module-enemy}}\label{\detokenize{enemy:enemy-module}}\label{\detokenize{enemy::doc}}\index{moduł@\spxentry{moduł}!enemy@\spxentry{enemy}}\index{enemy@\spxentry{enemy}!moduł@\spxentry{moduł}}
\sphinxAtStartPar
Moduł zawierający klasę węża przeciwnika „Enemy”, która dziedziczy po
klasie abstrakcyjnej węża „Snake”.
\index{Enemy (klasa w module enemy)@\spxentry{Enemy}\spxextra{klasa w module enemy}}

\begin{fulllineitems}
\phantomsection\label{\detokenize{enemy:enemy.Enemy}}
\pysigstartsignatures
\pysigline{\sphinxbfcode{\sphinxupquote{class\DUrole{w,w}{  }}}\sphinxbfcode{\sphinxupquote{Enemy}}}
\pysigstopsignatures
\sphinxAtStartPar
Klasy bazowe: {\hyperref[\detokenize{snake:snake.Snake}]{\sphinxcrossref{\sphinxcode{\sphinxupquote{Snake}}}}}

\sphinxAtStartPar
Klasa węża przeciwnika, dziedzicząca z klasy Snake.

\sphinxAtStartPar
Ustawianie celu poruszania się węże, na podstawie położenia innych węży
i jedzenia.
\index{\_\_ai\_update\_timer (Enemy atrybut)@\spxentry{\_\_ai\_update\_timer}\spxextra{Enemy atrybut}}

\begin{fulllineitems}
\phantomsection\label{\detokenize{enemy:enemy.Enemy.__ai_update_timer}}
\pysigstartsignatures
\pysigline{\sphinxbfcode{\sphinxupquote{\_\_ai\_update\_timer}}}
\pysigstopsignatures
\sphinxAtStartPar
Licznik czasu odświeżenia AI.
\begin{quote}\begin{description}
\sphinxlineitem{Type}
\sphinxAtStartPar
float

\end{description}\end{quote}

\end{fulllineitems}

\index{update() (Enemy metoda)@\spxentry{update()}\spxextra{Enemy metoda}}

\begin{fulllineitems}
\phantomsection\label{\detokenize{enemy:enemy.Enemy.update}}
\pysigstartsignatures
\pysiglinewithargsret{\sphinxbfcode{\sphinxupquote{update}}}{\sphinxparam{\DUrole{n,n}{food\_list}\DUrole{p,p}{:}\DUrole{w,w}{  }\DUrole{n,n}{list\DUrole{p,p}{{[}}{\hyperref[\detokenize{food:food.Food}]{\sphinxcrossref{food.Food}}}\DUrole{p,p}{{]}}}}, \sphinxparam{\DUrole{n,n}{enemies\_list}\DUrole{p,p}{:}\DUrole{w,w}{  }\DUrole{n,n}{list\DUrole{p,p}{{[}}{\hyperref[\detokenize{snake:snake.Snake}]{\sphinxcrossref{snake.Snake}}}\DUrole{p,p}{{]}}}}, \sphinxparam{\DUrole{n,n}{dt}\DUrole{p,p}{:}\DUrole{w,w}{  }\DUrole{n,n}{float}}}{}
\pysigstopsignatures
\sphinxAtStartPar
Odświeżanie AI i wybieranie pozycji celu poruszania się węża.
\begin{quote}\begin{description}
\sphinxlineitem{Parametry}\begin{itemize}
\item {} 
\sphinxAtStartPar
\sphinxstyleliteralstrong{\sphinxupquote{food\_list}} \textendash{} Lista obiektów jedzenia.

\item {} 
\sphinxAtStartPar
\sphinxstyleliteralstrong{\sphinxupquote{enemies\_list}} \textendash{} Lista obiektów węży.

\item {} 
\sphinxAtStartPar
\sphinxstyleliteralstrong{\sphinxupquote{dt}} \textendash{} Mnożnik zmieniający wartości względem wydajności gry.

\end{itemize}

\end{description}\end{quote}

\end{fulllineitems}


\end{fulllineitems}


\sphinxstepscope


\section{food module}
\label{\detokenize{food:module-food}}\label{\detokenize{food:food-module}}\label{\detokenize{food::doc}}\index{moduł@\spxentry{moduł}!food@\spxentry{food}}\index{food@\spxentry{food}!moduł@\spxentry{moduł}}
\sphinxAtStartPar
Moduł zawierający klasę jedzenia „Food”.
\index{Food (klasa w module food)@\spxentry{Food}\spxextra{klasa w module food}}

\begin{fulllineitems}
\phantomsection\label{\detokenize{food:food.Food}}
\pysigstartsignatures
\pysiglinewithargsret{\sphinxbfcode{\sphinxupquote{class\DUrole{w,w}{  }}}\sphinxbfcode{\sphinxupquote{Food}}}{\sphinxparam{\DUrole{n,n}{pos}\DUrole{p,p}{:}\DUrole{w,w}{  }\DUrole{n,n}{Vector2}\DUrole{w,w}{  }\DUrole{o,o}{=}\DUrole{w,w}{  }\DUrole{default_value}{None}}}{}
\pysigstopsignatures
\sphinxAtStartPar
Klasy bazowe: \sphinxcode{\sphinxupquote{object}}

\sphinxAtStartPar
Klasa jedzenia.
\index{pos (Food atrybut)@\spxentry{pos}\spxextra{Food atrybut}}

\begin{fulllineitems}
\phantomsection\label{\detokenize{food:food.Food.pos}}
\pysigstartsignatures
\pysigline{\sphinxbfcode{\sphinxupquote{pos}}}
\pysigstopsignatures
\sphinxAtStartPar
Pozycja jedzenia na mapie.
\begin{quote}\begin{description}
\sphinxlineitem{Type}
\sphinxAtStartPar
Vector2

\end{description}\end{quote}

\end{fulllineitems}

\index{color (Food atrybut)@\spxentry{color}\spxextra{Food atrybut}}

\begin{fulllineitems}
\phantomsection\label{\detokenize{food:food.Food.color}}
\pysigstartsignatures
\pysigline{\sphinxbfcode{\sphinxupquote{color}}}
\pysigstopsignatures
\sphinxAtStartPar
Kolor jedzenia.
\begin{quote}\begin{description}
\sphinxlineitem{Type}
\sphinxAtStartPar
Color

\end{description}\end{quote}

\end{fulllineitems}

\index{spawn\_timer (Food atrybut)@\spxentry{spawn\_timer}\spxextra{Food atrybut}}

\begin{fulllineitems}
\phantomsection\label{\detokenize{food:food.Food.spawn_timer}}
\pysigstartsignatures
\pysigline{\sphinxbfcode{\sphinxupquote{spawn\_timer}}}
\pysigstopsignatures
\sphinxAtStartPar
Czas pozostały do pojawienia się jedzenia.
\begin{quote}\begin{description}
\sphinxlineitem{Type}
\sphinxAtStartPar
float

\end{description}\end{quote}

\end{fulllineitems}

\index{update() (Food metoda)@\spxentry{update()}\spxextra{Food metoda}}

\begin{fulllineitems}
\phantomsection\label{\detokenize{food:food.Food.update}}
\pysigstartsignatures
\pysiglinewithargsret{\sphinxbfcode{\sphinxupquote{update}}}{\sphinxparam{\DUrole{n,n}{dt}\DUrole{p,p}{:}\DUrole{w,w}{  }\DUrole{n,n}{float}}}{}
\pysigstopsignatures
\sphinxAtStartPar
Odświeżanie koloru jedzenia.
\begin{quote}\begin{description}
\sphinxlineitem{Parametry}
\sphinxAtStartPar
\sphinxstyleliteralstrong{\sphinxupquote{dt}} \textendash{} Mnożnik zmieniający wartości względem wydajności gry.

\end{description}\end{quote}

\end{fulllineitems}


\end{fulllineitems}


\sphinxstepscope


\section{main module}
\label{\detokenize{main:module-main}}\label{\detokenize{main:main-module}}\label{\detokenize{main::doc}}\index{moduł@\spxentry{moduł}!main@\spxentry{main}}\index{main@\spxentry{main}!moduł@\spxentry{moduł}}
\sphinxAtStartPar
Główny moduł gry. Wywołuje pozostałe wszystkie moduły projektu. Uruchomienie
tego modułu skutkuje uruchomieniem gry.
\index{Main (klasa w module main)@\spxentry{Main}\spxextra{klasa w module main}}

\begin{fulllineitems}
\phantomsection\label{\detokenize{main:main.Main}}
\pysigstartsignatures
\pysigline{\sphinxbfcode{\sphinxupquote{class\DUrole{w,w}{  }}}\sphinxbfcode{\sphinxupquote{Main}}}
\pysigstopsignatures
\sphinxAtStartPar
Klasy bazowe: \sphinxcode{\sphinxupquote{object}}

\sphinxAtStartPar
Główna klasa programu. Odpowiada za przygotowanie elementów gry, obsługę
głównej pętli gry oraz rysowanie.
\index{screen (Main atrybut)@\spxentry{screen}\spxextra{Main atrybut}}

\begin{fulllineitems}
\phantomsection\label{\detokenize{main:main.Main.screen}}
\pysigstartsignatures
\pysigline{\sphinxbfcode{\sphinxupquote{screen}}}
\pysigstopsignatures
\sphinxAtStartPar
Płaszczyzna ekranu.
\begin{quote}\begin{description}
\sphinxlineitem{Type}
\sphinxAtStartPar
Surface

\end{description}\end{quote}

\end{fulllineitems}

\index{draw\_screen (Main atrybut)@\spxentry{draw\_screen}\spxextra{Main atrybut}}

\begin{fulllineitems}
\phantomsection\label{\detokenize{main:main.Main.draw_screen}}
\pysigstartsignatures
\pysigline{\sphinxbfcode{\sphinxupquote{draw\_screen}}}
\pysigstopsignatures
\sphinxAtStartPar
Płaszczyzna do rysowania.
\begin{quote}\begin{description}
\sphinxlineitem{Type}
\sphinxAtStartPar
Surface

\end{description}\end{quote}

\end{fulllineitems}

\index{draw\_screen\_size (Main atrybut)@\spxentry{draw\_screen\_size}\spxextra{Main atrybut}}

\begin{fulllineitems}
\phantomsection\label{\detokenize{main:main.Main.draw_screen_size}}
\pysigstartsignatures
\pysigline{\sphinxbfcode{\sphinxupquote{draw\_screen\_size}}}
\pysigstopsignatures
\sphinxAtStartPar
Wymiary płaszczyzny do rysowania.
\begin{quote}\begin{description}
\sphinxlineitem{Type}
\sphinxAtStartPar
Vector2

\end{description}\end{quote}

\end{fulllineitems}

\index{clock (Main atrybut)@\spxentry{clock}\spxextra{Main atrybut}}

\begin{fulllineitems}
\phantomsection\label{\detokenize{main:main.Main.clock}}
\pysigstartsignatures
\pysigline{\sphinxbfcode{\sphinxupquote{clock}}}
\pysigstopsignatures
\sphinxAtStartPar
Zegar gry.
\begin{quote}\begin{description}
\sphinxlineitem{Type}
\sphinxAtStartPar
Clock

\end{description}\end{quote}

\end{fulllineitems}

\index{dt (Main atrybut)@\spxentry{dt}\spxextra{Main atrybut}}

\begin{fulllineitems}
\phantomsection\label{\detokenize{main:main.Main.dt}}
\pysigstartsignatures
\pysigline{\sphinxbfcode{\sphinxupquote{dt}}}
\pysigstopsignatures
\sphinxAtStartPar
Mnożnik zmieniający wartości względem wydajności gry.
\begin{quote}\begin{description}
\sphinxlineitem{Type}
\sphinxAtStartPar
float

\end{description}\end{quote}

\end{fulllineitems}

\index{mouse (Main atrybut)@\spxentry{mouse}\spxextra{Main atrybut}}

\begin{fulllineitems}
\phantomsection\label{\detokenize{main:main.Main.mouse}}
\pysigstartsignatures
\pysigline{\sphinxbfcode{\sphinxupquote{mouse}}}
\pysigstopsignatures
\sphinxAtStartPar
Obiekt myszki.
\begin{quote}\begin{description}
\sphinxlineitem{Type}
\sphinxAtStartPar
{\hyperref[\detokenize{mouse:mouse.Mouse}]{\sphinxcrossref{Mouse}}}

\end{description}\end{quote}

\end{fulllineitems}

\index{is\_running (Main atrybut)@\spxentry{is\_running}\spxextra{Main atrybut}}

\begin{fulllineitems}
\phantomsection\label{\detokenize{main:main.Main.is_running}}
\pysigstartsignatures
\pysigline{\sphinxbfcode{\sphinxupquote{is\_running}}}
\pysigstopsignatures
\sphinxAtStartPar
Zmienna kontrolująca, czy gra ma być uruchomiona.
\begin{quote}\begin{description}
\sphinxlineitem{Type}
\sphinxAtStartPar
bool

\end{description}\end{quote}

\end{fulllineitems}

\index{scroll (Main atrybut)@\spxentry{scroll}\spxextra{Main atrybut}}

\begin{fulllineitems}
\phantomsection\label{\detokenize{main:main.Main.scroll}}
\pysigstartsignatures
\pysigline{\sphinxbfcode{\sphinxupquote{scroll}}}
\pysigstopsignatures
\sphinxAtStartPar
Przesunięcie obszaru rysowania gry.
\begin{quote}\begin{description}
\sphinxlineitem{Type}
\sphinxAtStartPar
Vector2

\end{description}\end{quote}

\end{fulllineitems}

\index{particles (Main atrybut)@\spxentry{particles}\spxextra{Main atrybut}}

\begin{fulllineitems}
\phantomsection\label{\detokenize{main:main.Main.particles}}
\pysigstartsignatures
\pysigline{\sphinxbfcode{\sphinxupquote{particles}}}
\pysigstopsignatures
\sphinxAtStartPar
Lista obiektów cząsteczek.
\begin{quote}\begin{description}
\sphinxlineitem{Type}
\sphinxAtStartPar
list{[}{\hyperref[\detokenize{particle:particle.Particle}]{\sphinxcrossref{Particle}}}{]}

\end{description}\end{quote}

\end{fulllineitems}

\index{player (Main atrybut)@\spxentry{player}\spxextra{Main atrybut}}

\begin{fulllineitems}
\phantomsection\label{\detokenize{main:main.Main.player}}
\pysigstartsignatures
\pysigline{\sphinxbfcode{\sphinxupquote{player}}}
\pysigstopsignatures
\sphinxAtStartPar
Obiekt węża gracza.
\begin{quote}\begin{description}
\sphinxlineitem{Type}
\sphinxAtStartPar
{\hyperref[\detokenize{player:player.Player}]{\sphinxcrossref{Player}}}

\end{description}\end{quote}

\end{fulllineitems}

\index{enemies (Main atrybut)@\spxentry{enemies}\spxextra{Main atrybut}}

\begin{fulllineitems}
\phantomsection\label{\detokenize{main:main.Main.enemies}}
\pysigstartsignatures
\pysigline{\sphinxbfcode{\sphinxupquote{enemies}}}
\pysigstopsignatures
\sphinxAtStartPar
Lista obiektów przeciwników.
\begin{quote}\begin{description}
\sphinxlineitem{Type}
\sphinxAtStartPar
list{[}Enemies{]}

\end{description}\end{quote}

\end{fulllineitems}

\index{food (Main atrybut)@\spxentry{food}\spxextra{Main atrybut}}

\begin{fulllineitems}
\phantomsection\label{\detokenize{main:main.Main.food}}
\pysigstartsignatures
\pysigline{\sphinxbfcode{\sphinxupquote{food}}}
\pysigstopsignatures
\sphinxAtStartPar
Lista obiektów jedzenia.
\begin{quote}\begin{description}
\sphinxlineitem{Type}
\sphinxAtStartPar
list{[}{\hyperref[\detokenize{food:food.Food}]{\sphinxcrossref{Food}}}{]}

\end{description}\end{quote}

\end{fulllineitems}

\index{add\_food() (Main metoda)@\spxentry{add\_food()}\spxextra{Main metoda}}

\begin{fulllineitems}
\phantomsection\label{\detokenize{main:main.Main.add_food}}
\pysigstartsignatures
\pysiglinewithargsret{\sphinxbfcode{\sphinxupquote{add\_food}}}{}{}
\pysigstopsignatures
\sphinxAtStartPar
Dodanie nowego obiektu jedzenia, po wcześniejszym znalezieniu
wolnej pozycji.

\end{fulllineitems}

\index{check\_snake\_collisions() (Main metoda statyczna)@\spxentry{check\_snake\_collisions()}\spxextra{Main metoda statyczna}}

\begin{fulllineitems}
\phantomsection\label{\detokenize{main:main.Main.check_snake_collisions}}
\pysigstartsignatures
\pysiglinewithargsret{\sphinxbfcode{\sphinxupquote{static\DUrole{w,w}{  }}}\sphinxbfcode{\sphinxupquote{check\_snake\_collisions}}}{\sphinxparam{\DUrole{n,n}{snake}\DUrole{p,p}{:}\DUrole{w,w}{  }\DUrole{n,n}{{\hyperref[\detokenize{snake:snake.Snake}]{\sphinxcrossref{Snake}}}}}, \sphinxparam{\DUrole{n,n}{other\_snakes}\DUrole{p,p}{:}\DUrole{w,w}{  }\DUrole{n,n}{list\DUrole{p,p}{{[}}{\hyperref[\detokenize{snake:snake.Snake}]{\sphinxcrossref{snake.Snake}}}\DUrole{p,p}{{]}}}}}{{ $\rightarrow$ bool}}
\pysigstopsignatures
\sphinxAtStartPar
Funkcja sprawdzająca kolizje podanego węża z innymi wężami.
\begin{quote}\begin{description}
\sphinxlineitem{Parametry}\begin{itemize}
\item {} 
\sphinxAtStartPar
\sphinxstyleliteralstrong{\sphinxupquote{snake}} \textendash{} Wąż, którego kolizje są sprawdzane.

\item {} 
\sphinxAtStartPar
\sphinxstyleliteralstrong{\sphinxupquote{other\_snakes}} \textendash{} Lista pozostałych węży.

\end{itemize}

\sphinxlineitem{Zwraca}
\sphinxAtStartPar
Prawda, jeżeli wystąpiła kolizja, fałsz w przeciwnym przypadku.

\end{description}\end{quote}

\end{fulllineitems}

\index{display\_update() (Main metoda)@\spxentry{display\_update()}\spxextra{Main metoda}}

\begin{fulllineitems}
\phantomsection\label{\detokenize{main:main.Main.display_update}}
\pysigstartsignatures
\pysiglinewithargsret{\sphinxbfcode{\sphinxupquote{display\_update}}}{}{}
\pysigstopsignatures
\sphinxAtStartPar
Odświeżanie ekranu.

\end{fulllineitems}

\index{draw() (Main metoda)@\spxentry{draw()}\spxextra{Main metoda}}

\begin{fulllineitems}
\phantomsection\label{\detokenize{main:main.Main.draw}}
\pysigstartsignatures
\pysiglinewithargsret{\sphinxbfcode{\sphinxupquote{draw}}}{}{}
\pysigstopsignatures
\sphinxAtStartPar
Rysowanie wszystkich elementów gry.

\end{fulllineitems}

\index{draw\_circle() (Main metoda)@\spxentry{draw\_circle()}\spxextra{Main metoda}}

\begin{fulllineitems}
\phantomsection\label{\detokenize{main:main.Main.draw_circle}}
\pysigstartsignatures
\pysiglinewithargsret{\sphinxbfcode{\sphinxupquote{draw\_circle}}}{\sphinxparam{\DUrole{n,n}{pos}\DUrole{p,p}{:}\DUrole{w,w}{  }\DUrole{n,n}{Vector2}}, \sphinxparam{\DUrole{n,n}{r}\DUrole{p,p}{:}\DUrole{w,w}{  }\DUrole{n,n}{float}}, \sphinxparam{\DUrole{n,n}{c}\DUrole{p,p}{:}\DUrole{w,w}{  }\DUrole{n,n}{Color}}, \sphinxparam{\DUrole{o,o}{**}\DUrole{n,n}{kwargs}}}{}
\pysigstopsignatures
\sphinxAtStartPar
Rysowanie koła.
\begin{quote}\begin{description}
\sphinxlineitem{Parametry}\begin{itemize}
\item {} 
\sphinxAtStartPar
\sphinxstyleliteralstrong{\sphinxupquote{pos}} \textendash{} Pozycja środka koła.

\item {} 
\sphinxAtStartPar
\sphinxstyleliteralstrong{\sphinxupquote{r}} \textendash{} Promień koła.

\item {} 
\sphinxAtStartPar
\sphinxstyleliteralstrong{\sphinxupquote{c}} \textendash{} Kolor koła.

\item {} 
\sphinxAtStartPar
\sphinxstyleliteralstrong{\sphinxupquote{**kwargs}} \textendash{} Dodatkowe argumenty rysowania.

\end{itemize}

\end{description}\end{quote}

\end{fulllineitems}

\index{draw\_eye() (Main metoda)@\spxentry{draw\_eye()}\spxextra{Main metoda}}

\begin{fulllineitems}
\phantomsection\label{\detokenize{main:main.Main.draw_eye}}
\pysigstartsignatures
\pysiglinewithargsret{\sphinxbfcode{\sphinxupquote{draw\_eye}}}{\sphinxparam{\DUrole{n,n}{pos}\DUrole{p,p}{:}\DUrole{w,w}{  }\DUrole{n,n}{Vector2}}, \sphinxparam{\DUrole{n,n}{size}\DUrole{p,p}{:}\DUrole{w,w}{  }\DUrole{n,n}{float}}, \sphinxparam{\DUrole{n,n}{transparency}}}{}
\pysigstopsignatures
\sphinxAtStartPar
Rysowanie oka węża.
\begin{quote}\begin{description}
\sphinxlineitem{Parametry}\begin{itemize}
\item {} 
\sphinxAtStartPar
\sphinxstyleliteralstrong{\sphinxupquote{pos}} \textendash{} Pozycja, na której ma zostać narysowane oko.

\item {} 
\sphinxAtStartPar
\sphinxstyleliteralstrong{\sphinxupquote{size}} \textendash{} Wielkość oka.

\item {} 
\sphinxAtStartPar
\sphinxstyleliteralstrong{\sphinxupquote{transparency}} (\sphinxstyleliteralemphasis{\sphinxupquote{int}}) \textendash{} Przeźroczystość oka.

\end{itemize}

\end{description}\end{quote}

\end{fulllineitems}

\index{draw\_snake() (Main metoda)@\spxentry{draw\_snake()}\spxextra{Main metoda}}

\begin{fulllineitems}
\phantomsection\label{\detokenize{main:main.Main.draw_snake}}
\pysigstartsignatures
\pysiglinewithargsret{\sphinxbfcode{\sphinxupquote{draw\_snake}}}{\sphinxparam{\DUrole{n,n}{snake}\DUrole{p,p}{:}\DUrole{w,w}{  }\DUrole{n,n}{{\hyperref[\detokenize{snake:snake.Snake}]{\sphinxcrossref{Snake}}}}}}{}
\pysigstopsignatures
\sphinxAtStartPar
Rysowanie węża.
\begin{quote}\begin{description}
\sphinxlineitem{Parametry}
\sphinxAtStartPar
\sphinxstyleliteralstrong{\sphinxupquote{snake}} \textendash{} Obiekt węża do narysowania.

\end{description}\end{quote}

\end{fulllineitems}

\index{events\_update() (Main metoda)@\spxentry{events\_update()}\spxextra{Main metoda}}

\begin{fulllineitems}
\phantomsection\label{\detokenize{main:main.Main.events_update}}
\pysigstartsignatures
\pysiglinewithargsret{\sphinxbfcode{\sphinxupquote{events\_update}}}{}{}
\pysigstopsignatures
\sphinxAtStartPar
Sprawdzanie, czy wywołano zdarzenie z modułu pygame i wykonanie
powiązanych z nim poleceń.

\end{fulllineitems}

\index{game() (Main metoda)@\spxentry{game()}\spxextra{Main metoda}}

\begin{fulllineitems}
\phantomsection\label{\detokenize{main:main.Main.game}}
\pysigstartsignatures
\pysiglinewithargsret{\sphinxbfcode{\sphinxupquote{game}}}{}{}
\pysigstopsignatures
\sphinxAtStartPar
Główna logika gry

\end{fulllineitems}

\index{is\_in\_screen\_range() (Main metoda)@\spxentry{is\_in\_screen\_range()}\spxextra{Main metoda}}

\begin{fulllineitems}
\phantomsection\label{\detokenize{main:main.Main.is_in_screen_range}}
\pysigstartsignatures
\pysiglinewithargsret{\sphinxbfcode{\sphinxupquote{is\_in\_screen\_range}}}{\sphinxparam{\DUrole{n,n}{vector}\DUrole{p,p}{:}\DUrole{w,w}{  }\DUrole{n,n}{Vector2}}, \sphinxparam{\DUrole{n,n}{shift}\DUrole{p,p}{:}\DUrole{w,w}{  }\DUrole{n,n}{float}\DUrole{w,w}{  }\DUrole{o,o}{=}\DUrole{w,w}{  }\DUrole{default_value}{0}}}{{ $\rightarrow$ bool}}
\pysigstopsignatures
\sphinxAtStartPar
Sprawdzanie, czy okrąg o podanej pozycji środka i promieniu znajduje
się w zasięgu rysowania.
\begin{quote}\begin{description}
\sphinxlineitem{Parametry}\begin{itemize}
\item {} 
\sphinxAtStartPar
\sphinxstyleliteralstrong{\sphinxupquote{vector}} \textendash{} Środek okręgu.

\item {} 
\sphinxAtStartPar
\sphinxstyleliteralstrong{\sphinxupquote{shift}} \textendash{} Promień okręgu.

\end{itemize}

\sphinxlineitem{Zwraca}
\sphinxAtStartPar
Prawda, jeżeli okrąg jest w zasięgu rysowania, fałsz w przeciwnym
przypadku.

\end{description}\end{quote}

\end{fulllineitems}

\index{rotate\_point() (Main metoda statyczna)@\spxentry{rotate\_point()}\spxextra{Main metoda statyczna}}

\begin{fulllineitems}
\phantomsection\label{\detokenize{main:main.Main.rotate_point}}
\pysigstartsignatures
\pysiglinewithargsret{\sphinxbfcode{\sphinxupquote{static\DUrole{w,w}{  }}}\sphinxbfcode{\sphinxupquote{rotate\_point}}}{\sphinxparam{\DUrole{n,n}{v1}\DUrole{p,p}{:}\DUrole{w,w}{  }\DUrole{n,n}{Vector2}}, \sphinxparam{\DUrole{n,n}{v2}\DUrole{p,p}{:}\DUrole{w,w}{  }\DUrole{n,n}{Vector2}}, \sphinxparam{\DUrole{n,n}{angle}\DUrole{p,p}{:}\DUrole{w,w}{  }\DUrole{n,n}{float}}}{}
\pysigstopsignatures
\sphinxAtStartPar
Funkcja zwracająca punkt wektor będący wektorem v1 obróconym o
podany kąt wokół wektora v2.
\begin{quote}\begin{description}
\sphinxlineitem{Parametry}\begin{itemize}
\item {} 
\sphinxAtStartPar
\sphinxstyleliteralstrong{\sphinxupquote{v1}} \textendash{} Wektor, który będzie obracany.

\item {} 
\sphinxAtStartPar
\sphinxstyleliteralstrong{\sphinxupquote{v2}} \textendash{} Wektor, wokół którego wektor v1 będzie obracany.

\item {} 
\sphinxAtStartPar
\sphinxstyleliteralstrong{\sphinxupquote{angle}} \textendash{} Kąt, o jaki wektor v1 będzie obracany wokół wektora v2.

\end{itemize}

\sphinxlineitem{Zwraca}
\sphinxAtStartPar
Wektor, będący wektorem v1 obróconym o kąt angle, wokół wektora v2.

\end{description}\end{quote}

\end{fulllineitems}


\end{fulllineitems}


\sphinxstepscope


\section{mouse module}
\label{\detokenize{mouse:module-mouse}}\label{\detokenize{mouse:mouse-module}}\label{\detokenize{mouse::doc}}\index{moduł@\spxentry{moduł}!mouse@\spxentry{mouse}}\index{mouse@\spxentry{mouse}!moduł@\spxentry{moduł}}
\sphinxAtStartPar
Moduł zawierający klasę obsługującą mysz „Mouse”.
\index{Mouse (klasa w module mouse)@\spxentry{Mouse}\spxextra{klasa w module mouse}}

\begin{fulllineitems}
\phantomsection\label{\detokenize{mouse:mouse.Mouse}}
\pysigstartsignatures
\pysigline{\sphinxbfcode{\sphinxupquote{class\DUrole{w,w}{  }}}\sphinxbfcode{\sphinxupquote{Mouse}}}
\pysigstopsignatures
\sphinxAtStartPar
Klasy bazowe: \sphinxcode{\sphinxupquote{object}}

\sphinxAtStartPar
Klasa obsługująca myszkę.
\index{clicked (Mouse atrybut)@\spxentry{clicked}\spxextra{Mouse atrybut}}

\begin{fulllineitems}
\phantomsection\label{\detokenize{mouse:mouse.Mouse.clicked}}
\pysigstartsignatures
\pysigline{\sphinxbfcode{\sphinxupquote{clicked}}}
\pysigstopsignatures
\sphinxAtStartPar
Zmienna mówiąca, czy lewy przycisk myszy jest wciśnięty.
\begin{quote}\begin{description}
\sphinxlineitem{Type}
\sphinxAtStartPar
bool

\end{description}\end{quote}

\end{fulllineitems}

\index{pos (Mouse atrybut)@\spxentry{pos}\spxextra{Mouse atrybut}}

\begin{fulllineitems}
\phantomsection\label{\detokenize{mouse:mouse.Mouse.pos}}
\pysigstartsignatures
\pysigline{\sphinxbfcode{\sphinxupquote{pos}}}
\pysigstopsignatures
\sphinxAtStartPar
Pozycja kursora myszy.
\begin{quote}\begin{description}
\sphinxlineitem{Type}
\sphinxAtStartPar
Vector2

\end{description}\end{quote}

\end{fulllineitems}

\index{update() (Mouse metoda)@\spxentry{update()}\spxextra{Mouse metoda}}

\begin{fulllineitems}
\phantomsection\label{\detokenize{mouse:mouse.Mouse.update}}
\pysigstartsignatures
\pysiglinewithargsret{\sphinxbfcode{\sphinxupquote{update}}}{\sphinxparam{\DUrole{n,n}{draw\_screen\_size}\DUrole{p,p}{:}\DUrole{w,w}{  }\DUrole{n,n}{Vector2}}}{}
\pysigstopsignatures
\sphinxAtStartPar
Odświeżania pozycji myszy i informacji o wciśnięciu przycisku.
\begin{quote}\begin{description}
\sphinxlineitem{Parametry}
\sphinxAtStartPar
\sphinxstyleliteralstrong{\sphinxupquote{draw\_screen\_size}} \textendash{} Aktualna wielkość płaszczyzny do rysowania.

\end{description}\end{quote}

\end{fulllineitems}


\end{fulllineitems}


\sphinxstepscope


\section{particle module}
\label{\detokenize{particle:module-particle}}\label{\detokenize{particle:particle-module}}\label{\detokenize{particle::doc}}\index{moduł@\spxentry{moduł}!particle@\spxentry{particle}}\index{particle@\spxentry{particle}!moduł@\spxentry{moduł}}
\sphinxAtStartPar
Moduł zawierający klasę cząsteczki „Particle”, wykorzystywaną podczas
tworzenia efektów cząsteczkowych.
\index{Particle (klasa w module particle)@\spxentry{Particle}\spxextra{klasa w module particle}}

\begin{fulllineitems}
\phantomsection\label{\detokenize{particle:particle.Particle}}
\pysigstartsignatures
\pysiglinewithargsret{\sphinxbfcode{\sphinxupquote{class\DUrole{w,w}{  }}}\sphinxbfcode{\sphinxupquote{Particle}}}{\sphinxparam{\DUrole{n,n}{pos}\DUrole{p,p}{:}\DUrole{w,w}{  }\DUrole{n,n}{Vector2}}}{}
\pysigstopsignatures
\sphinxAtStartPar
Klasy bazowe: \sphinxcode{\sphinxupquote{object}}

\sphinxAtStartPar
Klasa cząsteczki, służących do wyświetlania efektów cząsteczkowych.
\index{pos (Particle atrybut)@\spxentry{pos}\spxextra{Particle atrybut}}

\begin{fulllineitems}
\phantomsection\label{\detokenize{particle:particle.Particle.pos}}
\pysigstartsignatures
\pysigline{\sphinxbfcode{\sphinxupquote{pos}}}
\pysigstopsignatures
\sphinxAtStartPar
Pozycja cząsteczki.
\begin{quote}\begin{description}
\sphinxlineitem{Type}
\sphinxAtStartPar
Vector2

\end{description}\end{quote}

\end{fulllineitems}

\index{color (Particle atrybut)@\spxentry{color}\spxextra{Particle atrybut}}

\begin{fulllineitems}
\phantomsection\label{\detokenize{particle:particle.Particle.color}}
\pysigstartsignatures
\pysigline{\sphinxbfcode{\sphinxupquote{color}}}
\pysigstopsignatures
\sphinxAtStartPar
Kolor cząsteczki.
\begin{quote}\begin{description}
\sphinxlineitem{Type}
\sphinxAtStartPar
Color

\end{description}\end{quote}

\end{fulllineitems}

\index{size (Particle atrybut)@\spxentry{size}\spxextra{Particle atrybut}}

\begin{fulllineitems}
\phantomsection\label{\detokenize{particle:particle.Particle.size}}
\pysigstartsignatures
\pysigline{\sphinxbfcode{\sphinxupquote{size}}}
\pysigstopsignatures
\sphinxAtStartPar
Wielkość cząsteczki.
\begin{quote}\begin{description}
\sphinxlineitem{Type}
\sphinxAtStartPar
int

\end{description}\end{quote}

\end{fulllineitems}

\index{timer (Particle atrybut)@\spxentry{timer}\spxextra{Particle atrybut}}

\begin{fulllineitems}
\phantomsection\label{\detokenize{particle:particle.Particle.timer}}
\pysigstartsignatures
\pysigline{\sphinxbfcode{\sphinxupquote{timer}}}
\pysigstopsignatures
\sphinxAtStartPar
Czas pozostały do zniknięcia cząsteczki.
\begin{quote}\begin{description}
\sphinxlineitem{Type}
\sphinxAtStartPar
float

\end{description}\end{quote}

\end{fulllineitems}

\index{direction (Particle atrybut)@\spxentry{direction}\spxextra{Particle atrybut}}

\begin{fulllineitems}
\phantomsection\label{\detokenize{particle:particle.Particle.direction}}
\pysigstartsignatures
\pysigline{\sphinxbfcode{\sphinxupquote{direction}}}
\pysigstopsignatures
\sphinxAtStartPar
Kierunek poruszania się cząsteczki.
\begin{quote}\begin{description}
\sphinxlineitem{Type}
\sphinxAtStartPar
Vector2

\end{description}\end{quote}

\end{fulllineitems}

\index{update() (Particle metoda)@\spxentry{update()}\spxextra{Particle metoda}}

\begin{fulllineitems}
\phantomsection\label{\detokenize{particle:particle.Particle.update}}
\pysigstartsignatures
\pysiglinewithargsret{\sphinxbfcode{\sphinxupquote{update}}}{\sphinxparam{\DUrole{n,n}{dt}\DUrole{p,p}{:}\DUrole{w,w}{  }\DUrole{n,n}{float}}}{}
\pysigstopsignatures
\sphinxAtStartPar
Odświeżanie pozycji i przezroczystości koloru cząsteczki.
\begin{quote}\begin{description}
\sphinxlineitem{Parametry}
\sphinxAtStartPar
\sphinxstyleliteralstrong{\sphinxupquote{dt}} \textendash{} Mnożnik zmieniający wartości względem wydajności gry.

\end{description}\end{quote}

\end{fulllineitems}


\end{fulllineitems}


\sphinxstepscope


\section{player module}
\label{\detokenize{player:module-player}}\label{\detokenize{player:player-module}}\label{\detokenize{player::doc}}\index{moduł@\spxentry{moduł}!player@\spxentry{player}}\index{player@\spxentry{player}!moduł@\spxentry{moduł}}
\sphinxAtStartPar
Moduł zawierający klasę węża gracza „Player”, która dziedziczy po klasie
abstrakcyjnej węża „Snake”.
\index{Player (klasa w module player)@\spxentry{Player}\spxextra{klasa w module player}}

\begin{fulllineitems}
\phantomsection\label{\detokenize{player:player.Player}}
\pysigstartsignatures
\pysigline{\sphinxbfcode{\sphinxupquote{class\DUrole{w,w}{  }}}\sphinxbfcode{\sphinxupquote{Player}}}
\pysigstopsignatures
\sphinxAtStartPar
Klasy bazowe: {\hyperref[\detokenize{snake:snake.Snake}]{\sphinxcrossref{\sphinxcode{\sphinxupquote{Snake}}}}}

\sphinxAtStartPar
Klasa węża gracza, dziedzicząca z klasy Snake.

\sphinxAtStartPar
Ustawianie celu poruszania się węże, na podstawie wskaźnika myszki.
\index{update() (Player metoda)@\spxentry{update()}\spxextra{Player metoda}}

\begin{fulllineitems}
\phantomsection\label{\detokenize{player:player.Player.update}}
\pysigstartsignatures
\pysiglinewithargsret{\sphinxbfcode{\sphinxupquote{update}}}{\sphinxparam{\DUrole{n,n}{mouse}\DUrole{p,p}{:}\DUrole{w,w}{  }\DUrole{n,n}{{\hyperref[\detokenize{mouse:mouse.Mouse}]{\sphinxcrossref{Mouse}}}}}, \sphinxparam{\DUrole{n,n}{dt}\DUrole{p,p}{:}\DUrole{w,w}{  }\DUrole{n,n}{float}}, \sphinxparam{\DUrole{n,n}{scroll}\DUrole{p,p}{:}\DUrole{w,w}{  }\DUrole{n,n}{Vector2}}}{}
\pysigstopsignatures
\sphinxAtStartPar
Ustawienie kierunku poruszania na pozycję wskaźnika myszki
\begin{quote}\begin{description}
\sphinxlineitem{Parametry}\begin{itemize}
\item {} 
\sphinxAtStartPar
\sphinxstyleliteralstrong{\sphinxupquote{mouse}} \textendash{} Obiekt myszki

\item {} 
\sphinxAtStartPar
\sphinxstyleliteralstrong{\sphinxupquote{dt}} \textendash{} Mnożnik zmieniający wartości względem wydajności gry.

\item {} 
\sphinxAtStartPar
\sphinxstyleliteralstrong{\sphinxupquote{scroll}} \textendash{} Przesunięcie płaszczyzny do rysowania.

\end{itemize}

\end{description}\end{quote}

\end{fulllineitems}


\end{fulllineitems}


\sphinxstepscope


\section{snake module}
\label{\detokenize{snake:module-snake}}\label{\detokenize{snake:snake-module}}\label{\detokenize{snake::doc}}\index{moduł@\spxentry{moduł}!snake@\spxentry{snake}}\index{snake@\spxentry{snake}!moduł@\spxentry{moduł}}
\sphinxAtStartPar
Moduł zawierający abstrakcyjną klasę węża „Snake”, z której dziedziczą klasy
gracza „Player” oraz przeciwnika „Enemy”.
\index{Snake (klasa w module snake)@\spxentry{Snake}\spxextra{klasa w module snake}}

\begin{fulllineitems}
\phantomsection\label{\detokenize{snake:snake.Snake}}
\pysigstartsignatures
\pysigline{\sphinxbfcode{\sphinxupquote{class\DUrole{w,w}{  }}}\sphinxbfcode{\sphinxupquote{Snake}}}
\pysigstopsignatures
\sphinxAtStartPar
Klasy bazowe: \sphinxcode{\sphinxupquote{object}}

\sphinxAtStartPar
Abstrakcyjna klasa węża.

\sphinxAtStartPar
Zmiana pozycji, wielkości i koloru węża.
\index{eaten\_food (Snake atrybut)@\spxentry{eaten\_food}\spxextra{Snake atrybut}}

\begin{fulllineitems}
\phantomsection\label{\detokenize{snake:snake.Snake.eaten_food}}
\pysigstartsignatures
\pysigline{\sphinxbfcode{\sphinxupquote{eaten\_food}}}
\pysigstopsignatures
\sphinxAtStartPar
Ilość zjedzonego jedzenia do tej pory.
\begin{quote}\begin{description}
\sphinxlineitem{Type}
\sphinxAtStartPar
int

\end{description}\end{quote}

\end{fulllineitems}

\index{body\_size (Snake atrybut)@\spxentry{body\_size}\spxextra{Snake atrybut}}

\begin{fulllineitems}
\phantomsection\label{\detokenize{snake:snake.Snake.body_size}}
\pysigstartsignatures
\pysigline{\sphinxbfcode{\sphinxupquote{body\_size}}}
\pysigstopsignatures
\sphinxAtStartPar
Promień pojedynczej „komórki” ciała.
\begin{quote}\begin{description}
\sphinxlineitem{Type}
\sphinxAtStartPar
float

\end{description}\end{quote}

\end{fulllineitems}

\index{dest (Snake atrybut)@\spxentry{dest}\spxextra{Snake atrybut}}

\begin{fulllineitems}
\phantomsection\label{\detokenize{snake:snake.Snake.dest}}
\pysigstartsignatures
\pysigline{\sphinxbfcode{\sphinxupquote{dest}}}
\pysigstopsignatures
\sphinxAtStartPar
Pozycja, w kierunku której porusza się wąż.
\begin{quote}\begin{description}
\sphinxlineitem{Type}
\sphinxAtStartPar
Vector2

\end{description}\end{quote}

\end{fulllineitems}

\index{speed (Snake atrybut)@\spxentry{speed}\spxextra{Snake atrybut}}

\begin{fulllineitems}
\phantomsection\label{\detokenize{snake:snake.Snake.speed}}
\pysigstartsignatures
\pysigline{\sphinxbfcode{\sphinxupquote{speed}}}
\pysigstopsignatures
\sphinxAtStartPar
Prędkość węża.
\begin{quote}\begin{description}
\sphinxlineitem{Type}
\sphinxAtStartPar
float

\end{description}\end{quote}

\end{fulllineitems}

\index{is\_speeding (Snake atrybut)@\spxentry{is\_speeding}\spxextra{Snake atrybut}}

\begin{fulllineitems}
\phantomsection\label{\detokenize{snake:snake.Snake.is_speeding}}
\pysigstartsignatures
\pysigline{\sphinxbfcode{\sphinxupquote{is\_speeding}}}
\pysigstopsignatures
\sphinxAtStartPar
Stan przyśpieszania gracza.
\begin{quote}\begin{description}
\sphinxlineitem{Type}
\sphinxAtStartPar
bool

\end{description}\end{quote}

\end{fulllineitems}

\index{spawn\_timer (Snake atrybut)@\spxentry{spawn\_timer}\spxextra{Snake atrybut}}

\begin{fulllineitems}
\phantomsection\label{\detokenize{snake:snake.Snake.spawn_timer}}
\pysigstartsignatures
\pysigline{\sphinxbfcode{\sphinxupquote{spawn\_timer}}}
\pysigstopsignatures
\sphinxAtStartPar
Licznik czasu, jaki pozostał do pojawienia się węża
\begin{quote}\begin{description}
\sphinxlineitem{Type}
\sphinxAtStartPar
float

\end{description}\end{quote}

\end{fulllineitems}

\index{color (Snake atrybut)@\spxentry{color}\spxextra{Snake atrybut}}

\begin{fulllineitems}
\phantomsection\label{\detokenize{snake:snake.Snake.color}}
\pysigstartsignatures
\pysigline{\sphinxbfcode{\sphinxupquote{color}}}
\pysigstopsignatures
\sphinxAtStartPar
Kolor węża
\begin{quote}\begin{description}
\sphinxlineitem{Type}
\sphinxAtStartPar
Color

\end{description}\end{quote}

\end{fulllineitems}

\index{grow() (Snake metoda)@\spxentry{grow()}\spxextra{Snake metoda}}

\begin{fulllineitems}
\phantomsection\label{\detokenize{snake:snake.Snake.grow}}
\pysigstartsignatures
\pysiglinewithargsret{\sphinxbfcode{\sphinxupquote{grow}}}{}{}
\pysigstopsignatures
\sphinxAtStartPar
Zwiększenie ilości zjedzonego jedzenia i aktualizacja wielkości węża.

\end{fulllineitems}

\index{update\_snake() (Snake metoda)@\spxentry{update\_snake()}\spxextra{Snake metoda}}

\begin{fulllineitems}
\phantomsection\label{\detokenize{snake:snake.Snake.update_snake}}
\pysigstartsignatures
\pysiglinewithargsret{\sphinxbfcode{\sphinxupquote{update\_snake}}}{\sphinxparam{\DUrole{n,n}{dt}\DUrole{p,p}{:}\DUrole{w,w}{  }\DUrole{n,n}{float}}}{}
\pysigstopsignatures
\sphinxAtStartPar
Aktualizacja węża.
\begin{quote}\begin{description}
\sphinxlineitem{Parametry}
\sphinxAtStartPar
\sphinxstyleliteralstrong{\sphinxupquote{dt}} \textendash{} Mnożnik zmieniający wartości względem wydajności gry.

\end{description}\end{quote}

\end{fulllineitems}


\end{fulllineitems}



\chapter{Indices and tables}
\label{\detokenize{index:indices-and-tables}}\begin{itemize}
\item {} 
\sphinxAtStartPar
\DUrole{xref,std,std-ref}{genindex}

\item {} 
\sphinxAtStartPar
\DUrole{xref,std,std-ref}{modindex}

\item {} 
\sphinxAtStartPar
\DUrole{xref,std,std-ref}{search}

\end{itemize}


\renewcommand{\indexname}{Indeks modułów Pythona}
\begin{sphinxtheindex}
\let\bigletter\sphinxstyleindexlettergroup
\bigletter{c}
\item\relax\sphinxstyleindexentry{const}\sphinxstyleindexpageref{const:\detokenize{module-const}}
\indexspace
\bigletter{e}
\item\relax\sphinxstyleindexentry{enemy}\sphinxstyleindexpageref{enemy:\detokenize{module-enemy}}
\indexspace
\bigletter{f}
\item\relax\sphinxstyleindexentry{food}\sphinxstyleindexpageref{food:\detokenize{module-food}}
\indexspace
\bigletter{m}
\item\relax\sphinxstyleindexentry{main}\sphinxstyleindexpageref{main:\detokenize{module-main}}
\item\relax\sphinxstyleindexentry{mouse}\sphinxstyleindexpageref{mouse:\detokenize{module-mouse}}
\indexspace
\bigletter{p}
\item\relax\sphinxstyleindexentry{particle}\sphinxstyleindexpageref{particle:\detokenize{module-particle}}
\item\relax\sphinxstyleindexentry{player}\sphinxstyleindexpageref{player:\detokenize{module-player}}
\indexspace
\bigletter{s}
\item\relax\sphinxstyleindexentry{snake}\sphinxstyleindexpageref{snake:\detokenize{module-snake}}
\end{sphinxtheindex}

\renewcommand{\indexname}{Indeks}
\printindex
\end{document}